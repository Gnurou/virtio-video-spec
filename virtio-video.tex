\section{Video Device}\label{sec:Device Types / Video Device}

The virtio video encoder and decoder devices provide support for
host-accelerated video encoding and decoding. Despite being different
devices types, they use the same protocol and general flow.

\subsection{Device ID}\label{sec:Device Types / Video Device / Device ID}

\begin{description}
\item[30]
encoder device
\item[31]
decoder device
\end{description}

\subsection{Virtqueues}\label{sec:Device Types / Video Device / Virtqueues}

\begin{description}
\item[0]
commandq - queue for driver commands and device responses to these
commands.
\item[1]
eventq - queue for events sent by the device to the driver.
\end{description}

\subsection{Feature bits}\label{sec:Device Types / Video Device / Feature bits}

\begin{description}
\item[VIRTIO\_VIDEO\_F\_RESOURCE\_GUEST\_PAGES (0)]
Guest pages can be used as the backing memory of resources.
\item[VIRTIO\_VIDEO\_F\_RESOURCE\_NON\_CONTIG (1)]
The device can use non-contiguous guest memory as the backing memory of
resources. Only meaningful if VIRTIO\_VIDEO\_F\_RESOURCE\_GUEST\_PAGES
is also set.
\item[VIRTIO\_VIDEO\_F\_RESOURCE\_DYNAMIC (2)]
The device supports re-attaching memory to resources while streaming.
\item[VIRTIO\_VIDEO\_F\_RESOURCE\_VIRTIO\_OBJECT (3)]
Objects exported by another virtio device can be used as the backing
memory of resources.
\end{description}

\devicenormative{\subsubsection}{Feature bits}{Device Types / Video Device / Feature bits}

The device MUST present at least one of
VIRTIO\_VIDEO\_F\_RESOURCE\_GUEST\_PAGES or
VIRTIO\_VIDEO\_F\_RESOURCE\_VIRTIO\_OBJECT, since the absence of both
bits would mean that no memory can be used at all for resources.

\drivernormative{\subsubsection}{Feature bits}{Device Types / Video Device / Feature bits}

The driver MUST negotiate at least one of the
VIRTIO\_VIDEO\_F\_RESOURCE\_GUEST\_PAGES and
VIRTIO\_VIDEO\_F\_RESOURCE\_VIRTIO\_OBJECT features.

If the VIRTIO\_VIDEO\_F\_RESOURCE\_NON\_CONTIG is not present, the
driver MUST use physically contiguous memory for all the buffers it
allocates.

\subsection{Device configuration layout}\label{sec:Device Types / Video Device / Device configuration layout}

Video device configuration uses the following layout:

\begin{lstlisting}
struct virtio_video_config {
        le32 version;
        le32 caps_length;
};
\end{lstlisting}

\begin{description}
\item[\field{version}]
is the protocol version that the device understands.
\item[\field{caps_length}]
is the minumum length in bytes that a device-writable buffer must have
in order to receive the response to
VIRTIO\_VIDEO\_CMD\_DEVICE\_QUERY\_CAPS.
\end{description}

\devicenormative{\subsubsection}{Device configuration layout}{Device Types / Video Device / Device configuration layout}

As there is currently only one version of the protocol, the device MUST
set the \field{version} field to 0.

The device MUST set the \field{caps_length} field to a value equal to
the response size of VIRTIO\_VIDEO\_CMD\_DEVICE\_QUERY\_CAPS.

\subsection{Device Initialization}\label{sec:Device Types / Video Device / Device Initialization}

\begin{enumerate}
\def\labelenumi{\arabic{enumi}.}
\item
  The driver reads the feature bits and negotiates the features it
  needs.
\item
  The driver sets up the commandq.
\item
  The driver confirms that it supports the version of the protocol
  advertized in the \field{version} field of the configuration space.
\item
  The driver reads the \field{caps_length} field of the configuration
  space and prepares a buffer of at least that size.
\item
  The driver sends that buffer on the commandq with the
  VIRTIO\_VIDEO\_CMD\_DEVICE\_QUERY\_CAPS command.
\item
  The driver receives the reponse from the device, and parses its
  capabilities.
\end{enumerate}

\subsection{Device Operation}\label{sec:Device Types / Video Device / Device Operation}

The commandq is used by the driver to send commands to the device and to
receive the device's response via used buffers.

The driver can create new streams using the
VIRTIO\_VIDEO\_CMD\_STREAM\_CREATE command. Each stream has two resource
queues (not to be confused with the virtio queues) called INPUT and
OUTPUT. The INPUT queue accepts driver-filled input data for the device
(bitstream data for a decoder ; input frames for an encoder), while the
OUTPUT queue receives resources to be filled by the device as a result
of processing the INPUT queue (decoded frames for a decoder ; encoded
bitstream data for an encoder).

A resource is a set of memory buffers that contain a unit of data that
the device can process or produce. Most resources will only have one
buffer (like bitstreams and single-planar images), but frames using a
multi-planar format will have several.

Before resources can be submitted to a queue, backing memory must be
attached to them using VIRTIO\_VIDEO\_CMD\_RESOURCE\_ATTACH\_BACKING.
The exact form of that memory is negotiated using the feature flags.

The INPUT and OUTPUT queues are effectively independent, and the driver
can fill them without caring about the other queue. In particular there
is no need to queue input and output resources in pairs: one input
resource can result in zero to many output resources being produced.

Resources are queued to the INPUT or OUTPUT queue using the
VIRTIO\_VIDEO\_CMD\_RESOURCE\_QUEUE command. The device replies to this
command when an input resource has been fully processed and can be
reused by the driver, or when an output resource has been filled by the
device as a result of processing.

Parameters of the stream can be obtained and configured using
VIRTIO\_VIDEO\_CMD\_STREAM\_GET\_PARAM and
VIRTIO\_VIDEO\_CMD\_STREAM\_SET\_PARAM. Available parameters depend on
on the device type and are detailed in section
\ref{sec:Device Types / Video Device / Parameters}.

The device may detect stream-related events that require intervention
from the driver and signals them on the eventq. One example is a dynamic
resolution change while decoding a stream, which may require the driver
to reallocate the backing memory of its output resources to fit the new
resolution.

\drivernormative{\subsubsection}{Device Operation}{Device Types / Video Device / Device Operation}

Descriptor chains sent to the commandq by the driver MUST include at
least one device-writable descriptor of a size sufficient to receive the
response to the queued command.

\devicenormative{\subsubsection}{Device Operation}{Device Types / Video Device / Device Operation}

Responses to a command MUST be written by the device in the first
device-writable descriptor of the descriptor chain from which the
command came.

\subsubsection{Device Operation: Command Virtqueue}\label{sec:Device Types / Video Device / Device Operation / Device Operation: Command Virtqueue}

This section details the commands that can be sent on the commandq by
the driver, as well as the responses that the device will write.

Different structures are used for each command and response. A command
structure starts with the requested command code, defined as follows:

\begin{lstlisting}
/* Device */
#define VIRTIO_VIDEO_CMD_DEVICE_QUERY_CAPS       0x100

/* Stream */
#define VIRTIO_VIDEO_CMD_STREAM_CREATE           0x200
#define VIRTIO_VIDEO_CMD_STREAM_DESTROY          0x201
#define VIRTIO_VIDEO_CMD_STREAM_DRAIN            0x203
#define VIRTIO_VIDEO_CMD_STREAM_STOP             0x204
#define VIRTIO_VIDEO_CMD_STREAM_GET_PARAM        0x205
#define VIRTIO_VIDEO_CMD_STREAM_SET_PARAM        0x206

/* Resource*/
#define VIRTIO_VIDEO_CMD_RESOURCE_ATTACH_BACKING 0x400
#define VIRTIO_VIDEO_CMD_RESOURCE_QUEUE          0x401
};
\end{lstlisting}

A response structure starts with the result of the requested command,
defined as follows:

\begin{lstlisting}
/* Success */
#define VIRTIO_VIDEO_RESULT_OK                          0x000

/* Error */
#define VIRTIO_VIDEO_RESULT_ERR_INVALID_COMMAND         0x100
#define VIRTIO_VIDEO_RESULT_ERR_INVALID_OPERATION       0x101
#define VIRTIO_VIDEO_RESULT_ERR_INVALID_STREAM_ID       0x102
#define VIRTIO_VIDEO_RESULT_ERR_INVALID_RESOURCE_ID     0x103
#define VIRTIO_VIDEO_RESULT_ERR_INVALID_ARGUMENT        0x104
#define VIRTIO_VIDEO_RESULT_ERR_CANCELED                0x105
#define VIRTIO_VIDEO_RESULT_ERR_OUT_OF_MEMORY           0x106
\end{lstlisting}

For response structures carrying an error code, the rest of the
structure is considered invalid. Only response structures carrying
VIRTIO\_VIDEO\_RESULT\_OK shall be examined further by the driver.

\devicenormative{\paragraph}{Device Operation: Command Virtqueue}{Device Types / Video Device / Device Operation / Device Operation: Command Virtqueue}

The device MUST return VIRTIO\_VIDEO\_RESULT\_ERR\_INVALID\_COMMAND to
any non-existing command code.

\drivernormative{\paragraph}{Device Operation: Command Virtqueue}{Device Types / Video Device / Device Operation / Device Operation: Command Virtqueue}

The driver MUST NOT interpret the rest of a response which result is not
VIRTIO\_VIDEO\_RESULT\_OK.

\subsubsection{Device Operation: Device Commands}\label{sec:Device Types / Video Device / Device Operation / Device Operation: Device Commands}

Device capabilities are retrieved using the
VIRTIO\_VIDEO\_CMD\_DEVICE\_QUERY\_CAPS command, which returns arrays of
formats supported by the input and output queues.

\paragraph{VIRTIO_VIDEO_CMD_DEVICE_QUERY_CAPS}\label{sec:Device Types / Video Device / Device Operation / Device Operation: Device Commands / VIRTIO_VIDEO_CMD_DEVICE_QUERY_CAPS}

Retrieve device capabilities.

The driver sends this command with
\field{struct virtio_video_device_query_caps}:

\begin{lstlisting}
struct virtio_video_device_query_caps {
        le32 cmd_type; /* VIRTIO_VIDEO_CMD_DEVICE_QUERY_CAPS */
};
\end{lstlisting}

The device responds with
\field{struct virtio_video_device_query_caps_resp}:

\begin{lstlisting}
struct virtio_video_device_query_caps_resp {
        le32 result; /* VIRTIO_VIDEO_RESULT_* */
        le32 num_bitstream_formats;
        le32 num_image_formats;
        /**
         * Followed by
         * struct virtio_video_bitstream_format_desc bitstream_formats[num_bitstream_formats];
         */
        /**
         * Followed by
         * struct virtio_video_image_format_desc image_formats[num_image_formats]
         */
};
\end{lstlisting}

\begin{description}
\item[\field{result}]
is

\begin{description}
\item[VIRTIO\_VIDEO\_RESULT\_OK]
if the operation succeeded,
\item[VIRTIO\_VIDEO\_RESULT\_ERR\_OUT\_OF\_MEMORY]
if the descriptor was smaller than the defined \field{caps_length} in
the video device configuration.
\end{description}
\item[\field{num_bitstream_formats}]
is the number of supported bitstream formats.
\item[\field{num_image_formats}]
is the number of supported image formats.
\item[\field{bitstream_formats}]
is an array of size \field{num_bitstream_formats} containing the
supported encoded formats. These correspond to the formats that can be
set on the INPUT queue for a decoder, and on the OUTPUT queue for an
encoder. For a description of bitstream formats, see
\ref{sec:Device Types / Video Device / Supported formats / Bitstream formats}.
\item[\field{image_formats}]
is an array of size \field{num_image_formats} containing the supported
image formats. These correspond to the formats that can be set on the
OUTPUT queue for a decoder, and on the INPUT queue for an encoder. For a
description of image formats, see
\ref{sec:Device Types / Video Device / Supported formats / Image formats}.
\end{description}

\drivernormative{\subparagraph}{VIRTIO_VIDEO_CMD_DEVICE_QUERY_CAPS}{Device Types / Video Device / Device Operation / Device Operation: Device Commands / VIRTIO_VIDEO_CMD_DEVICE_QUERY_CAPS}

\field{cmd_type} MUST be set to VIRTIO\_VIDEO\_CMD\_DEVICE\_QUERY\_CAPS
by the driver.

\devicenormative{\subparagraph}{VIRTIO_VIDEO_CMD_DEVICE_QUERY_CAPS}{Device Types / Video Device / Device Operation / Device Operation: Device Commands / VIRTIO_VIDEO_CMD_DEVICE_QUERY_CAPS}

The device MUST write the two \field{bitstream_formats} and
\field{image_formats} arrays, of length \field{num_bitstream_formats}
and \field{num_image_formats}, respectively.

The total size of the response MUST be equal to \field{caps_length}
bytes, as reported by the device configuration.

\subsubsection{Device Operation: Stream commands}\label{sec:Device Types / Video Device / Device Operation / Device Operation: Stream commands}

Stream commands allow the creation, destruction, and flow control of a
stream.

\paragraph{VIRTIO_VIDEO_CMD_STREAM_CREATE}\label{sec:Device Types / Video Device / Device Operation / Device Operation: Stream commands / VIRTIO_VIDEO_CMD_STREAM_CREATE}

Create a new stream using the device.

The driver sends this command with
\field{struct virtio_video_stream_create}:

\begin{lstlisting}
struct virtio_video_stream_create {
        le32 cmd_type; /* VIRTIO_VIDEO_CMD_STREAM_CREATE */
};
\end{lstlisting}

The device responds with \field{struct virtio_video_stream_create_resp}:

\begin{lstlisting}
struct virtio_video_stream_create_resp {
        le32 result; /* VIRTIO_VIDEO_RESULT_* */
        le32 stream_id;
};
\end{lstlisting}

\begin{description}
\item[\field{result}]
is

\begin{description}
\item[VIRTIO\_VIDEO\_RESULT\_OK]
if the operation succeeded,
\item[VIRTIO\_VIDEO\_RESULT\_ERR\_OUT\_OF\_MEMORY]
if the limit of simultaneous streams has been reached by the device and
no more can be created.
\item[VIRTIO\_VIDEO\_RESULT\_ERR\_INVALID\_COMMAND]
if the stream cannot be created due to an unexpected device issue.
\end{description}
\item[\field{stream_id}]
is the ID of the created stream allocated by the device.
\end{description}

\drivernormative{\subparagraph}{VIRTIO_VIDEO_CMD_STREAM_CREATE}{Device Types / Video Device / Device Operation / Device Operation: Stream commands / VIRTIO_VIDEO_CMD_STREAM_CREATE}

\field{cmd_type} MUST be set to VIRTIO\_VIDEO\_CMD\_STREAM\_CREATE by
the driver.

\devicenormative{\subparagraph}{VIRTIO_VIDEO_CMD_STREAM_CREATE}{Device Types / Video Device / Device Operation / Device Operation: Stream commands / VIRTIO_VIDEO_CMD_STREAM_CREATE}

\field{stream_id} MUST be set to an identifier that is unique to that
stream for as long as it lives.

\paragraph{VIRTIO_VIDEO_CMD_STREAM_DESTROY}\label{sec:Device Types / Video Device / Device Operation / Device Operation: Stream commands / VIRTIO_VIDEO_CMD_STREAM_DESTROY}

Destroy a video stream and all its resources. Any activity on the stream
is halted and all resources released by the time the response is
received by the driver.

The driver sends this command with
\field{struct virtio_video_stream_destroy}:

\begin{lstlisting}
struct virtio_video_stream_destroy {
         le32 cmd_type; /* VIRTIO_VIDEO_CMD_STREAM_DESTROY */
         le32 stream_id;
};
\end{lstlisting}

\begin{description}
\item[\field{stream_id}]
is the ID of the stream to be destroyed, as previously returned by
VIRTIO\_VIDEO\_CMD\_STREAM\_CREATE.
\end{description}

The device responds with
\field{struct virtio_video_stream_destroy_resp}:

\begin{lstlisting}
struct virtio_video_stream_destroy_resp {
         le32 result; /* VIRTIO_VIDEO_RESULT_* */
};
\end{lstlisting}

\begin{description}
\item[\field{result}]
is

\begin{description}
\item[VIRTIO\_VIDEO\_RESULT\_OK]
if the operation succeeded,
\item[VIRTIO\_VIDEO\_RESULT\_ERR\_INVALID\_STREAM\_ID]
if the requested stream ID does not exist.
\end{description}
\end{description}

\drivernormative{\subparagraph}{VIRTIO_VIDEO_CMD_STREAM_DESTROY}{Device Types / Video Device / Device Operation / Device Operation: Stream commands / VIRTIO_VIDEO_CMD_STREAM_DESTROY}

\field{cmd_type} MUST be set to VIRTIO\_VIDEO\_CMD\_STREAM\_DESTROY by
the driver.

\field{stream_id} MUST be set to a valid stream ID previously returned
by VIRTIO\_VIDEO\_CMD\_STREAM\_CREATE.

The driver MUST stop using \field{stream_id} as a valid stream after it
received the response to this command.

\devicenormative{\subparagraph}{VIRTIO_VIDEO_CMD_STREAM_DESTROY}{Device Types / Video Device / Device Operation / Device Operation: Stream commands / VIRTIO_VIDEO_CMD_STREAM_DESTROY}

Before the device sends a response, it MUST respond with
VIRTIO\_VIDEO\_RESULT\_ERR\_CANCELED to all pending commands.

After responding to this command, the device MUST reply with
VIRTIO\_VIDEO\_RESULT\_ERR\_INVALID\_STREAM\_ID to any command related
to this stream.

\paragraph{VIRTIO_VIDEO_CMD_STREAM_DRAIN}\label{sec:Device Types / Video Device / Device Operation / Device Operation: Stream commands / VIRTIO_VIDEO_CMD_STREAM_DRAIN}

Complete processing of all currently queued input resources.

VIRTIO\_VIDEO\_CMD\_STREAM\_DRAIN ensures that all sent
VIRTIO\_VIDEO\_CMD\_RESOURCE\_QUEUE commands on the INPUT queue are
processed by the device and that the resulting output resources are
available to the driver.

The driver sends this command with
\field{struct virtio_video_stream_drain}:

\begin{lstlisting}
struct virtio_video_stream_drain {
        le32 cmd_type; /* VIRTIO_VIDEO_CMD_STREAM_DRAIN */
        le32 stream_id;
};
\end{lstlisting}

\begin{description}
\item[\field{stream_id}]
is the ID of the stream to drain, as previously returned by
VIRTIO\_VIDEO\_CMD\_STREAM\_CREATE.
\end{description}

The device responds with \field{struct virtio_video_stream_drain_resp}
once the drain operation is completed:

\begin{lstlisting}
struct virtio_video_stream_drain_resp {
        le32 result; /* VIRTIO_VIDEO_RESULT_* */
};
\end{lstlisting}

\begin{description}
\item[\field{result}]
is

\begin{description}
\item[VIRTIO\_VIDEO\_RESULT\_OK]
if the operation succeeded,
\item[VIRTIO\_VIDEO\_RESULT\_ERR\_INVALID\_STREAM\_ID]
if the requested stream does not exist,
\item[VIRTIO\_VIDEO\_RESULT\_ERR\_INVALID\_OPERATION]
if a drain operation is already in progress for this stream,
\item[VIRTIO\_VIDEO\_RESULT\_ERR\_CANCELED]
if the operation has been canceled by a VIRTIO\_VIDEO\_CMD\_STREAM\_STOP
or VIRTIO\_VIDEO\_CMD\_STREAM\_DESTROY operation.
\end{description}
\end{description}

\drivernormative{\subparagraph}{VIRTIO_VIDEO_CMD_STREAM_DRAIN}{Device Types / Video Device / Device Operation / Device Operation: Stream commands / VIRTIO_VIDEO_CMD_STREAM_DRAIN}

\field{cmd_type} MUST be set to VIRTIO\_VIDEO\_CMD\_STREAM\_DRAIN by the
driver.

\field{stream_id} MUST be set to a valid stream ID previously returned
by VIRTIO\_VIDEO\_CMD\_STREAM\_CREATE.

The driver MUST keep queueing output resources until it gets the
response to this command. Failure to do so may result in the device
stalling as it waits for output resources to write into.

The driver MUST account for the fact that the response to this command
might come out-of-order (i.e.~after other commands sent to the device),
and that it can be interrupted.

\devicenormative{\subparagraph}{VIRTIO_VIDEO_CMD_STREAM_DRAIN}{Device Types / Video Device / Device Operation / Device Operation: Stream commands / VIRTIO_VIDEO_CMD_STREAM_DRAIN}

Before the device sends the response, it MUST process and respond to all
the VIRTIO\_VIDEO\_CMD\_RESOURCE\_QUEUE commands on the INPUT queue that
were sent before the drain command, and make all the corresponding
output resources available to the driver by responding to their
VIRTIO\_VIDEO\_CMD\_RESOURCE\_QUEUE command.

While the device is processing the command, it MUST return
VIRTIO\_VIDEO\_RESULT\_ERR\_INVALID\_OPERATION to the
VIRTIO\_VIDEO\_CMD\_STREAM\_DRAIN command.

If the command is interrupted due to a VIRTIO\_VIDEO\_CMD\_STREAM\_STOP
or VIRTIO\_VIDEO\_CMD\_STREAM\_DESTROY operation, the device MUST
respond with VIRTIO\_VIDEO\_RESULT\_ERR\_CANCELED.

\paragraph{VIRTIO_VIDEO_CMD_STREAM_STOP}\label{sec:Device Types / Video Device / Device Operation / Device Operation: Stream commands / VIRTIO_VIDEO_CMD_STREAM_STOP}

Immediately return all queued input resources without processing them
and stop operation until new input resources are queued.

This command is mostly useful for decoders that need to quickly jump
from one point of the stream to another (i.e.~seeking), or in order to
stop processing as quickly as possible.

The driver sends this command with
\field{struct virtio_video_stream_stop}:

\begin{lstlisting}
struct virtio_video_stream_stop {
        le32 cmd_type; /* VIRTIO_VIDEO_CMD_STREAM_STOP */
        le32 stream_id;
};
\end{lstlisting}

\begin{description}
\item[\field{stream_id}]
is the ID of the stream to stop, as previously returned by
VIRTIO\_VIDEO\_CMD\_STREAM\_CREATE.
\end{description}

The device responds with \field{struct virtio_video_stream_stop_resp}
after the response for all previously queued input resources has been
sent:

\begin{lstlisting}
struct virtio_video_stream_stop_resp {
        le32 result; /* VIRTIO_VIDEO_RESULT_* */
};
\end{lstlisting}

\begin{description}
\item[\field{result}]
is

\begin{description}
\item[VIRTIO\_VIDEO\_RESULT\_OK]
if the operation succeeded,
\item[VIRTIO\_VIDEO\_RESULT\_ERR\_INVALID\_STREAM\_ID]
if the requested stream does not exist,
\end{description}
\end{description}

\drivernormative{\subparagraph}{VIRTIO_VIDEO_CMD_STREAM_STOP}{Device Types / Video Device / Device Operation / Device Operation: Stream commands / VIRTIO_VIDEO_CMD_STREAM_STOP}

\field{cmd_type} MUST be set to VIRTIO\_VIDEO\_CMD\_STREAM\_STOP by the
driver.

\field{stream_id} MUST be set to a valid stream ID previously returned
by VIRTIO\_VIDEO\_CMD\_STREAM\_CREATE.

Upon receiving the response to this command, the driver SHOULD process
(or drop) any output resource before resuming operation by queueing new
input resources.

Upon receiving the response to this command, the driver CAN modify the
\field{struct virtio_video_params_resources} parameter corresponding to
the INPUT queue, and subsequently attach new backing memory to the input
resources using the VIRTIO\_VIDEO\_CMD\_RESOURCE\_ATTACH\_BACKING
command.

\devicenormative{\subparagraph}{VIRTIO_VIDEO_CMD_STREAM_STOP}{Device Types / Video Device / Device Operation / Device Operation: Stream commands / VIRTIO_VIDEO_CMD_STREAM_STOP}

The device MUST return VIRTIO\_VIDEO\_RESULT\_ERR\_CANCELED to any
pending VIRTIO\_VIDEO\_CMD\_STREAM\_DRAIN and
VIRTIO\_VIDEO\_CMD\_RESOURCE\_QUEUE command on the INPUT queue before
responding to this command. Pending commands on the output queue are not
affected.

The device MUST interrupt operation as quickly as possible, and not be
dependent on output resources being queued by the driver.

Upon resuming processing, the device CAN skip input data until it finds
a point that allows it to resume operation properly (e.g.~until a
keyframe it found in the input stream of a decoder).

\paragraph{VIRTIO_VIDEO_CMD_STREAM_GET_PARAM}\label{sec:Device Types / Video Device / Device Operation / Device Operation: Stream commands / VIRTIO_VIDEO_CMD_STREAM_GET_PARAM}

Read the value of a parameter of the given stream. Available parameters
depend on the device type and are listed in
\ref{sec:Device Types / Video Device / Parameters}.

\begin{lstlisting}
struct virtio_video_stream_get_param {
        le32 cmd_type; /* VIRTIO_VIDEO_CMD_STREAM_GET_PARAM */
        le32 stream_id;
        le32 param_type; /* VIRTIO_VIDEO_PARAMS_* */
        u8 padding[4];
}
\end{lstlisting}

\begin{description}
\item[\field{stream_id}]
is the ID of the stream we want to get a parameter from.
\item[\field{param_type}]
is one of the VIRTIO\_VIDEO\_PARAMS\_* values indicating the parameter
we want to get.
\end{description}

The device responds with \field{struct virtio_video_stream_param_resp}:

\begin{lstlisting}
struct virtio_video_stream_param_resp {
        le32 result; /* VIRTIO_VIDEO_RESULT_* */
        union virtio_video_stream_params param;
};
\end{lstlisting}

\begin{description}
\item[\field{result}]
is

\begin{description}
\item[VIRTIO\_VIDEO\_RESULT\_OK]
if the operation succeeded,
\item[VIRTIO\_VIDEO\_RESULT\_ERR\_INVALID\_STREAM\_ID]
if the requested stream does not exist,
\item[VIRTIO\_VIDEO\_RESULT\_ERR\_INVALID\_ARGUMENT]
if the \field{param_type} argument is invalid for the device,
\end{description}
\item[\field{param}]
is the value of the requested parameter, if \field{result} is
VIRTIO\_VIDEO\_RESULT\_OK.
\end{description}

\drivernormative{\subparagraph}{VIRTIO_VIDEO_CMD_STREAM_GET_PARAM}{Device Types / Video Device / Device Operation / Device Operation: Stream commands / VIRTIO_VIDEO_CMD_STREAM_GET_PARAM}

\field{cmd_type} MUST be set to VIRTIO\_VIDEO\_CMD\_STREAM\_GET\_PARAM
by the driver.

\field{stream_id} MUST be set to a valid stream ID previously returned
by VIRTIO\_VIDEO\_CMD\_STREAM\_CREATE.

\field{param_type} MUST be set to a parameter type that is valid for the
device.

\paragraph{VIRTIO_VIDEO_CMD_STREAM_SET_PARAM}\label{sec:Device Types / Video Device / Device Operation / Device Operation: Stream commands / VIRTIO_VIDEO_CMD_STREAM_SET_PARAM}

Write the value of a parameter of the given stream, and return the value
actually set by the device. Available parameters depend on the device
type and are listed in
\ref{sec:Device Types / Video Device / Parameters}.

\begin{lstlisting}
struct virtio_video_stream_set_param {
        le32 cmd_type; /* VIRTIO_VIDEO_CMD_STREAM_SET_PARAM */
        le32 stream_id;
        le32 param_type; /* VIRTIO_VIDEO_PARAMS_* */
        u8 padding[4];
        union virtio_video_stream_params param;
}
\end{lstlisting}

\begin{description}
\item[\field{stream_id}]
is the ID of the stream we want to set a parameter for.
\item[\field{param_type}]
is one of the VIRTIO\_VIDEO\_PARAMS\_* values indicating the parameter
we want to set.
\end{description}

The device responds with \field{struct virtio_video_stream_param_resp}:

\begin{lstlisting}
struct virtio_video_stream_param_resp {
        le32 result; /* VIRTIO_VIDEO_RESULT_* */
        union virtio_video_stream_params param;
};
\end{lstlisting}

\begin{description}
\item[\field{result}]
is

\begin{description}
\item[VIRTIO\_VIDEO\_RESULT\_OK]
if the operation succeeded,
\item[VIRTIO\_VIDEO\_RESULT\_ERR\_INVALID\_STREAM\_ID]
if the requested stream does not exist,
\item[VIRTIO\_VIDEO\_RESULT\_ERR\_INVALID\_ARGUMENT]
if the \field{param_type} argument is invalid for the device,
\item[VIRTIO\_VIDEO\_RESULT\_ERR\_INVALID\_OPERATION]
if the requested parameter cannot be modified at this moment.
\end{description}
\item[\field{param}]
is the actual value of the parameter set by the device, if
\field{result} is VIRTIO\_VIDEO\_RESULT\_OK. The value set by the device
may differ from the requested value depending on the device's
capabilities.
\end{description}

Outside of the error cases described above, setting a parameter does not
fail. If the device cannot apply the parameter as requested, it will
adjust it to the closest setting it supports, and return that value to
the driver. It is then up to the driver to decide whether it can work
within the range of parameters supported by the device.

\drivernormative{\subparagraph}{VIRTIO_VIDEO_CMD_STREAM_SET_PARAM}{Device Types / Video Device / Device Operation / Device Operation: Stream commands / VIRTIO_VIDEO_CMD_STREAM_SET_PARAM}

\field{cmd_type} MUST be set to VIRTIO\_VIDEO\_CMD\_STREAM\_SET\_PARAM
by the driver.

\field{stream_id} MUST be set to a valid stream ID previously returned
by VIRTIO\_VIDEO\_CMD\_STREAM\_CREATE.

\field{param_type} MUST be set to a parameter type that is valid for the
device, and \field{param} MUST be filled as the union member
corresponding to \field{param_type}.

The driver MUST check the actual value of the parameter as set by the
device and work with this value, or fail properly if it cannot.

\devicenormative{\subparagraph}{VIRTIO_VIDEO_CMD_STREAM_SET_PARAM}{Device Types / Video Device / Device Operation / Device Operation: Stream commands / VIRTIO_VIDEO_CMD_STREAM_SET_PARAM}

The device MUST NOT return an error if the value requested by the driver
cannot be applied as-is. Instead, the device MUST set the parameter to
the closest supported value to the one requested by the driver.

\subsubsection{Device Operation: Resource Commands}\label{sec:Device Types / Video Device / Device Operation / Device Operation: Resource Commands}

Resource commands manage the memory backing of individual resources and
allow to queue them so the device can process them.

\paragraph{VIRTIO_VIDEO_CMD_RESOURCE_ATTACH_BACKING}\label{sec:Device Types / Video Device / Device Operation / Device Operation: Resource Commands / VIRTIO_VIDEO_CMD_RESOURCE_ATTACH_BACKING}

Assign backing memory to a resource.

The driver sends this command with
\field{struct virtio_video_resource_attach_backing}:

\begin{lstlisting}
#define VIRTIO_VIDEO_QUEUE_TYPE_INPUT       0
#define VIRTIO_VIDEO_QUEUE_TYPE_OUTPUT      1

struct virtio_video_resource_attach_backing {
        le32 cmd_type; /* VIRTIO_VIDEO_CMD_RESOURCE_ATTACH_BACKING */
        le32 stream_id;
        le32 queue_type; /* VIRTIO_VIDEO_QUEUE_TYPE_* */
        le32 resource_id;
        union virtio_video_resource resources[];
};
\end{lstlisting}

\begin{description}
\item[\field{stream_id}]
is the ID of a valid stream.
\item[\field{queue_type}]
is the direction of the queue.
\item[\field{resource_id}]
is the ID of the resource to be attached to.
\item[\field{resources}]
specifies memory regions to attach.
\end{description}

The union \field{virtio_video_resource} is defined as follows:

\begin{lstlisting}
union virtio_video_resource {
        struct virtio_video_resource_sg_list sg_list;
        struct virtio_video_resource_object object;
};
\end{lstlisting}

\begin{description}
\item[\field{sg_list}]
represents a scatter-gather list. This variant can be used when the
\field{mem_type} member of the \field{virtio_video_params_resources}
corresponding to the queue is set to
VIRTIO\_VIDEO\_MEM\_TYPE\_GUEST\_PAGES (see
\ref{sec:Device Types / Video Device / Parameters / Common parameters}).
\item[\field{object}]
represents an object exported from another virtio device. This variant
can be used when the \field{mem_type} member of the
\field{virtio_video_params_resources} corresponding to the queue is set
to VIRTIO\_VIDEO\_MEM\_TYPE\_VIRTIO\_OBJECT (see
\ref{sec:Device Types / Video Device / Parameters / Common parameters}).
\end{description}

The struct \field{virtio_video_resource_sg_list} is defined as follows:

\begin{lstlisting}
struct virtio_video_resource_sg_entry {
        le64 addr;
        le32 length;
        u8 padding[4];
};

struct virtio_video_resource_sg_list {
        le32 num_entries;
        u8 padding[4];
        /* Followed by num_entries instances of
           video_video_resource_sg_entry */
};
\end{lstlisting}

Within \field{struct virtio_video_resource_sg_entry}:

\begin{description}
\item[\field{addr}]
is a guest physical address to the start of the SG entry.
\item[\field{length}]
is the length of the SG entry.
\end{description}

Finally, for \field{struct virtio_video_resource_sg_list}:

\begin{description}
\item[\field{num_entries}]
is the number of \field{struct virtio_video_resource_sg_entry} instances
that follow.
\end{description}

\field{struct virtio_video_resource_object} is defined as follows:

\begin{lstlisting}
struct virtio_video_resource_object {
        u8 uuid[16];
};
\end{lstlisting}

\begin{description}
\item[uuid]
is a version 4 UUID specified by \hyperref[intro:rfc4122]{[RFC4122]}.
\end{description}

The device responds with
\field{struct virtio_video_resource_attach_backing_resp}:

\begin{lstlisting}
struct virtio_video_resource_attach_backing_resp {
        le32 result; /* VIRTIO_VIDEO_RESULT_* */
};
\end{lstlisting}

\begin{description}
\item[\field{result}]
is

\begin{description}
\item[VIRTIO\_VIDEO\_RESULT\_OK]
if the operation succeeded,
\item[VIRTIO\_VIDEO\_RESULT\_ERR\_INVALID\_STREAM\_ID]
if the mentioned stream does not exist,
\item[VIRTIO\_VIDEO\_RESULT\_ERR\_INVALID\_ARGUMENT]
if \field{queue_type}, \field{resource_id}, or \field{resources} have an
invalid value,
\item[VIRTIO\_VIDEO\_RESULT\_ERR\_INVALID\_OPERATION]
if the operation is performed at a time when it is non-valid.
\end{description}
\end{description}

VIRTIO\_VIDEO\_CMD\_RESOURCE\_ATTACH\_BACKING can only be called during
the following times:

\begin{itemize}
\item
  AFTER a VIRTIO\_VIDEO\_CMD\_STREAM\_CREATE and BEFORE invoking
  VIRTIO\_VIDEO\_CMD\_RESOURCE\_QUEUE for the first time on the
  resource,
\item
  AFTER successfully changing the \field{virtio_video_params_resources}
  parameter corresponding to the queue and BEFORE
  VIRTIO\_VIDEO\_CMD\_RESOURCE\_QUEUE is called again on the resource.
\end{itemize}

This is to ensure that the device can rely on the fact that a given
resource will always point to the same memory for as long as it may be
used by the video device. For instance, a decoder may use returned
decoded frames as reference for future frames and won't overwrite the
backing resource of a frame that is being referenced. It is only before
a stream is started and after a Dynamic Resolution Change event has
occurred that we can be sure that all resources won't be used in that
way.

\drivernormative{\subparagraph}{VIRTIO_VIDEO_CMD_RESOURCE_ATTACH_BACKING}{Device Types / Video Device / Device Operation / Device Operation: Resource Commands / VIRTIO_VIDEO_CMD_RESOURCE_ATTACH_BACKING}

\field{cmd_type} MUST be set to
VIRTIO\_VIDEO\_CMD\_RESOURCE\_ATTACH\_BACKING by the driver.

\field{stream_id} MUST be set to a valid stream ID previously returned
by VIRTIO\_VIDEO\_CMD\_STREAM\_CREATE.

\field{queue_type} MUST be set to a valid queue type.

\field{resource_id} MUST be an integer inferior to the number of
resources currently allocated for the queue.

The length of the \field{resources} array of
\field{struct virtio_video_resource_attach_backing} MUST be equal to the
number of resources required by the format currently set on the queue,
as described in
\ref{sec:Device Types / Video Device / Supported formats}.

\devicenormative{\subparagraph}{VIRTIO_VIDEO_CMD_RESOURCE_ATTACH_BACKING}{Device Types / Video Device / Device Operation / Device Operation: Resource Commands / VIRTIO_VIDEO_CMD_RESOURCE_ATTACH_BACKING}

At any time other than the times valid for calling this command, the
device MUST return VIRTIO\_VIDEO\_RESULT\_ERR\_INVALID\_OPERATION.

\paragraph{VIRTIO_VIDEO_CMD_RESOURCE_QUEUE}\label{sec:Device Types / Video Device / Device Operation / Device Operation: Resource Commands / VIRTIO_VIDEO_CMD_RESOURCE_QUEUE}

Add a resource to the device's queue.

\begin{lstlisting}
#define VIRTIO_VIDEO_MAX_PLANES                    8

#define VIRTIO_VIDEO_ENQUEUE_FLAG_FORCE_KEY_FRAME  (1 << 0)

struct virtio_video_resource_queue {
        le32 cmd_type; /* VIRTIO_VIDEO_CMD_RESOURCE_ATTACH_BACKING */
        le32 stream_id;
        le32 queue_type; /* VIRTIO_VIDEO_QUEUE_TYPE_* */
        le32 resource_id;
        le32 flags; /* Bitmask of VIRTIO_VIDEO_ENQUEUE_FLAG_* */
        u8 padding[4];
        le64 timestamp;
        le32 data_sizes[VIRTIO_VIDEO_MAX_PLANES];
};
\end{lstlisting}

\begin{description}
\item[\field{stream_id}]
is the ID of a valid stream.
\item[\field{queue_type}]
is the direction of the queue.
\item[\field{resource_id}]
is the ID of the resource to be queued.
\item[\field{flags}]
is a bitmask of VIRTIO\_VIDEO\_ENQUEUE\_FLAG\_* values.

\begin{description}
\item[\field{VIRTIO_VIDEO_ENQUEUE_FLAG_FORCE_KEY_FRAME}]
The submitted frame is to be encoded as a key frame. Only valid for the
encoder's INPUT queue.
\end{description}
\item[\field{timestamp}]
is an abstract sequence counter that can be used on the INPUT queue for
synchronization. Resources produced on the output queue will carry the
\field{timestamp} of the input resource they have been produced from.
\item[\field{data_sizes}]
number of data bytes used for each plane. Set by the driver for input
resources.
\end{description}

The device responds with
\field{struct virtio_video_resource_queue_resp}:

\begin{lstlisting}
#define VIRTIO_VIDEO_DEQUEUE_FLAG_ERR           (1 << 0)
/* Encoder only */
#define VIRTIO_VIDEO_DEQUEUE_FLAG_KEY_FRAME     (1 << 1)
#define VIRTIO_VIDEO_DEQUEUE_FLAG_P_FRAME       (1 << 2)
#define VIRTIO_VIDEO_DEQUEUE_FLAG_B_FRAME       (1 << 3)

struct virtio_video_resource_queue_resp {
        le32 result;
        le32 flags;
        le64 timestamp;
        le32 data_sizes[VIRTIO_VIDEO_MAX_PLANES];
};
\end{lstlisting}

\begin{description}
\item[\field{result}]
is

\begin{description}
\item[VIRTIO\_VIDEO\_RESULT\_OK]
if the operation succeeded,
\item[VIRTIO\_VIDEO\_RESULT\_ERR\_INVALID\_STREAM\_ID]
if the requested stream does not exist,
\item[VIRTIO\_VIDEO\_RESULT\_ERR\_INVALID\_ARGUMENT]
if the \field{queue_type}, \field{resource_id} or \field{flags}
parameters have an invalid value,
\item[VIRTIO\_VIDEO\_RESULT\_ERR\_INVALID\_OPERATION]
if VIRTIO\_VIDEO\_CMD\_RESOURCE\_ATTACH\_BACKING has not been
successfully called on the resource prior to queueing it.
\item[VIRTIO\_VIDEO\_RESULT\_ERR\_CANCELED]
if the resource has not been processed, not because of an error but
because of a change in the state of the codec. The driver is expected to
take action and address the condition before submitting the resource
again.
\end{description}
\item[\field{flags}]
is a bitmask of VIRTIO\_VIDEO\_DEQUEUE\_FLAG\_* flags.

\begin{description}
\item[VIRTIO\_VIDEO\_DEQUEUE\_FLAG\_ERR]
is set on output resources when a non-fatal processing error has
happened and the data contained by the resource is likely to be
corrupted,
\item[VIRTIO\_VIDEO\_DEQUEUE\_FLAG\_KEY\_FRAME]
is set on output resources when the resource contains an encoded key
frame (only for encoders).
\item[VIRTIO\_VIDEO\_DEQUEUE\_FLAG\_P\_FRAME]
is set on output resources when the resource contains only differences
to a previous key frame (only for encoders).
\item[VIRTIO\_VIDEO\_DEQUEUE\_FLAG\_B\_FRAME]
is set on output resources when the resource contains only the
differences between the current frame and both the preceding and
following key frames (only for encoders).
\end{description}
\item[\field{timestamp}]
is set on output resources to the \field{timestamp} value of the input
resource that produced the resource.
\item[\field{data_sizes}]
is set on output resources to the amount of data written by the device,
for each plane.
\end{description}

\drivernormative{\subparagraph}{VIRTIO_VIDEO_CMD_RESOURCE_QUEUE}{Device Types / Video Device / Device Operation / Device Operation: Resource Commands / VIRTIO_VIDEO_CMD_RESOURCE_QUEUE}

\field{cmd_type} MUST be set to VIRTIO\_VIDEO\_CMD\_RESOURCE\_QUEUE by
the driver.

\field{stream_id} MUST be set to a valid stream ID previously returned
by VIRTIO\_VIDEO\_CMD\_STREAM\_CREATE.

\field{queue_type} MUST be set to a valid queue type.

\field{resource_id} MUST be an integer inferior to the number of
resources currently allocated for the queue.

The driver MUST account for the fact that the response to this command
might come out-of-order (i.e.~after other commands sent to the device),
and that it can be interrupted.

\devicenormative{\subparagraph}{VIRTIO_VIDEO_CMD_RESOURCE_QUEUE}{Device Types / Video Device / Device Operation / Device Operation: Resource Commands / VIRTIO_VIDEO_CMD_RESOURCE_QUEUE}

The device MUST mark output resources that might contain corrupted
content due to and error with the VIRTIO\_VIDEO\_BUFFER\_FLAG\_ERR flag.

For output resources, the device MUST copy the \field{timestamp}
parameter of the input resource that produced it into its response.

In case of encoder, the device MUST mark each output resource with one
of VIRTIO\_VIDEO\_DEQUEUE\_FLAG\_KEY\_FRAME,
VIRTIO\_VIDEO\_DEQUEUE\_FLAG\_P\_FRAME, or
VIRTIO\_VIDEO\_DEQUEUE\_FLAG\_B\_FRAME.

If the processing of a resource was stopped due to a stream event, a
VIRTIO\_VIDEO\_CMD\_STREAM\_STOP, or a
VIRTIO\_VIDEO\_CMD\_STREAM\_DESTROY, the device MUST set \field{result}
to VIRTIO\_VIDEO\_RESULT\_ERR\_CANCELED.

\subsubsection{Device Operation: Event Virtqueue}\label{sec:Device Types / Video Device / Device Operation / Device Operation: Event Virtqueue}

The eventq is used by the device to signal stream events that are not a
direct result of a command queued by the driver on the commandq. Since
these events affect the device operation, the driver is expected to
react to them and resume streaming afterwards.

There are currently two supported events: device error, and Dynamic
Resolution Change.

\begin{lstlisting}
#define VIRTIO_VIDEO_EVENT_ERROR    1
#define VIRTIO_VIDEO_EVENT_DRC      2

union virtio_video_event {
        le32 event_type /* One of VIRTIO_VIDEO_EVENT_* */
        struct virtio_video_event_err err;
        struct virtio_video_event_drc drc;
}
\end{lstlisting}

\drivernormative{\paragraph}{Device Operation: Event Virtqueue}{Device Types / Video Device / Device Operation / Device Operation: Event Virtqueue}

The driver MUST at any time have at least one descriptor with a used
buffer large enough to contain a \field{struct virtio_video_event}
queued on the eventq.

\paragraph{Error Event}\label{sec:Device Types / Video Device / Device Operation / Device Operation: Event Virtqueue / Error Event}

The error event is queued by the device when an unrecoverable error
occurred during processing. The stream is considered invalid from that
point and is automatically closed. Pending
VIRTIO\_VIDEO\_CMD\_STREAM\_DRAIN and
VIRTIO\_VIDEO\_CMD\_RESOURCE\_QUEUE commands are canceled, and further
commands will fail with VIRTIO\_VIDEO\_RESULT\_INVALID\_STREAM\_ID.

Note that this is different from dequeued resources carrying the
VIRTIO\_VIDEO\_DEQUEUE\_FLAG\_ERR flag. This flag indicates that the
output might be corrupted, but the stream in itself can continue and
might recover.

This event should only be used for catastrophic errors, e.g.~a host
driver failure that cannot be recovered.

\paragraph{Dynamic Resolution Change Event}\label{sec:Device Types / Video Device / Device Operation / Device Operation: Event Virtqueue / Dynamic Resolution Change Event}

A Dynamic Resolution Change (or DRC) event happens when a decoder device
detects that the resolution of the stream being decoded has changed.
This event is emitted after processing all the input resources preceding
the resolution change, and as a result all the output resources
corresponding to these pre-DRC input resources are available to the
driver by the time it receives the DRC event.

A DRC event automatically detaches the backing memory of all output
resources. Upon receiving the DRC event and processing all pending
output resources, the driver is responsible for querying the new output
resolution and re-attaching suitable backing memory to the output
resources before queueing them again. Streaming resumes when the first
output resource is queued with memory properly attached.

\devicenormative{\subparagraph}{Dynamic Resolution Change Event}{Device Types / Video Device / Device Operation / Device Operation: Event Virtqueue / Dynamic Resolution Change Event}

The device MUST make all the output resources that correspond to frames
before the resolution change point available to the driver BEFORE it
sends the resolution change event to the driver.

After the event is emitted, the device MUST reject all output resources
for which VIRTIO\_VIDEO\_CMD\_RESOURCE\_ATTACH\_BACKING has not been
successfully called again with
VIRTIO\_VIDEO\_RESULT\_ERR\_INVALID\_OPERATION.

\drivernormative{\subparagraph}{Dynamic Resolution Change Event}{Device Types / Video Device / Device Operation / Device Operation: Event Virtqueue / Dynamic Resolution Change Event}

The driver MUST query the new output resolution parameter and call
VIRTIO\_VIDEO\_CMD\_RESOURCE\_ATTACH\_BACKING with suitable memory for
each output resource before queueing them again.

\subsection{Parameters}\label{sec:Device Types / Video Device / Parameters}

Parameters allow the driver to configure the device for the decoding or
encoding operation.

The \field{union virtio_video_stream_params} is defined as follows:

\begin{lstlisting}
/* Common parameters */
#define VIRTIO_VIDEO_PARAMS_INPUT_RESOURCES             0x001
#define VIRTIO_VIDEO_PARAMS_OUTPUT_RESOURCES            0x002

/* Decoder-only parameters */
#define VIRTIO_VIDEO_PARAMS_DECODER_INPUT_FORMAT        0x101
#define VIRTIO_VIDEO_PARAMS_DECODER_OUTPUT_FORMAT       0x102

/* Encoder-only parameters */
#define VIRTIO_VIDEO_PARAMS_ENCODER_INPUT_FORMAT        0x201
#define VIRTIO_VIDEO_PARAMS_ENCODER_OUTPUT_FORMAT       0x202
#define VIRTIO_VIDEO_PARAMS_ENCODER_BITRATE             0x203

union virtio_video_stream_params {
        /* Common parameters */
        struct virtio_video_params_resources input_resources;
        struct virtio_video_params_resources output_resources;

        /* Decoder-only parameters */
        struct virtio_video_params_bitstream_format decoder_input_format;
        struct virtio_video_params_image_format decoder_output_format;

        /* Encoder-only parameters */
        struct virtio_video_params_image_format encoder_input_format;
        struct virtio_video_params_bitstream_format encoder_output_format;
        struct virtio_video_params_bitrate encoder_bitrate;
};
\end{lstlisting}

Not all parameters are valid for all devices. For instance, a decoder
does not support any of the encoder-only parameters and will return
VIRTIO\_VIDEO\_RESULT\_ERR\_INVALID\_OPERATION if an unsupported
parameter is queried or set.

Each parameter is described in the remainder of this section.

\drivernormative{\subsubsection}{Parameters}{Device Types / Video Device / Parameters}

After creating a new stream, the initial value of all parameters is
undefined to the driver. Thus, the driver MUST NOT assume the default
value of any parameter and MUST use
VIRTIO\_VIDEO\_CMD\_STREAM\_GET\_PARAM in order to get the values of the
parameters is needs.

The driver SHOULD modify parameters by first calling
VIRTIO\_VIDEO\_CMD\_STREAM\_GET\_PARAM to get the current value of the
parameter it wants to modify, alter it and submit the desired value
using VIRTIO\_VIDEO\_CMD\_STREAM\_SET\_PARAM, then checking the actual
value set to the parameter in the response of
VIRTIO\_VIDEO\_CMD\_STREAM\_SET\_PARAM.

\devicenormative{\subsubsection}{Parameters}{Device Types / Video Device / Parameters}

The device MUST initialize each parameter to a valid default value and
allow each parameter to be read even without the driver explicitly
setting a value for it.

\subsubsection{Common parameters}\label{sec:Device Types / Video Device / Parameters / Common parameters}

\field{struct virtio_video_params_resources} is used to control the
number of resources and their backing memory type for the INPUT and
OUTPUT queues:

\begin{lstlisting}
#define VIRTIO_VIDEO_MEM_TYPE_GUEST_PAGES       0x1
#define VIRTIO_VIDEO_MEM_TYPE_VIRTIO_OBJECT     0x2

struct virtio_video_params_resources {
        le32 min_resources;
        le32 max_resources;
        le32 num_resources;
        u8 mem_type; /* VIRTIO_VIDEO_MEM_TYPE_* */
        u8 padding[3];
}
\end{lstlisting}

\begin{description}
\item[\field{min_resources}]
is the minimum number of resources that the queue supports for the
current settings. Cannot be modified by the driver.
\item[\field{max_resources}]
is the maximum number of resources that the queue supports for the
current settings. Cannot be modified by the driver.
\item[\field{num_resources}]
is the number of resources that can be addressed for the queue, numbered
from \(0\) to \(num\_queue - 1\). Can be equal to zero if no resources
are allocated, otherwise will be comprised between \field{min_resources}
and \field{max_resources}.
\item[\field{mem_type}]
is the memory type that will be used to back these resources.
\end{description}

Successfully setting this parameter results in all currently attached
resources of the corresponding queue to become detached, i.e.~the driver
cannot queue a resource to the queue without attaching some backing
memory first. All currently queued resources for the queue are returned
with the VIRTIO\_VIDEO\_RESULT\_ERR\_CANCELED error before the response
to the VIRTIO\_VIDEO\_CMD\_STREAM\_SET\_PARAM is returned.

This parameter can only be changed during the following times:

\begin{itemize}
\item
  After creating a stream and before queuing any resource on a given
  queue,
\item
  For the INPUT queue, after receiving the reponse to a
  VIRTIO\_VIDEO\_CMD\_STREAM\_STOP and before queueing any input
  resource,
\item
  For the OUTPUT queue, after receiving a DRC event and before queueing
  any output resource.
\end{itemize}

Attempts to change this parameter outside of these times will result in
VIRTIO\_VIDEO\_RESULT\_ERR\_INVALID\_OPERATION to be returned.

\subsubsection{Format parameters}\label{sec:Device Types / Video Device / Parameters / Format parameters}

The format of the input and output queues are defined using the
\field{virtio_video_params_bitstream_format} and
\field{virtio_video_params_image_format}. Which one applies to the input
or output queue depends on whether the device is a decoder or an
encoder.

Bitstream formats are set using the
\field{virtio_video_params_bitstream_format} struct:

\begin{lstlisting}
struct virtio_video_params_bitstream_format {
        u8 fourcc[4];
        le32 buffer_size;
}
\end{lstlisting}

\begin{description}
\item[\field{fourcc}]
is the fourcc of the bitstream format. For a list of supported formats,
see
\ref{sec:Device Types / Video Device / Supported formats / Bitstream formats}.
\item[\field{buffer_size}]
is the minimum size of the buffers that will back resources to be
queued.
\end{description}

Image formats are set using the \field{virtio_video_params_image_format}
struct:

\begin{lstlisting}
struct virtio_video_rect {
        le32 left;
        le32 top;
        le32 width;
        le32 height;
}

struct virtio_video_plane_format {
        le32 buffer_size;
        le32 stride;
        le32 offset;
        u8 padding[4];
}

struct virtio_video_params_image_format {
        u8 fourcc[4];
        le32 width;
        le32 height;
        u8 padding[4];
        struct virtio_video_rect crop;
        struct virtio_video_plane_format planes[VIRTIO_VIDEO_MAX_PLANES];
}
\end{lstlisting}

\begin{description}
\item[\field{fourcc}]
is the fourcc of the image format. For a list of supported formats, see
\ref{sec:Device Types / Video Device / Supported formats / Image formats}.
\item[\field{width}]
is the width in pixels of the coded image.
\item[\field{height}]
is the height in pixels of the coded image.
\item[\field{crop}]
is the rectangle covering the visible size of the frame, i.e the part of
the frame that should be displayed.
\item[\field{planes}]
is the format description of each individual plane making this format.
The number of planes is dependent on the \field{fourcc} and detailed in
\ref{sec:Device Types / Video Device / Supported formats / Image formats}.

\begin{description}
\item[\field{buffer_size}]
is the minimum size of the buffers that will back resources to be
queued.
\item[\field{stride}]
is the distance in bytes between two lines of data.
\item[\field{offset}]
is the starting offset for the data in the buffer.
\end{description}
\end{description}

\devicenormative{\paragraph}{Format parameters}{Device Types / Video Device / Parameters / Format parameters}

The device MAY adjust any requested parameter to the nearest-supported
value if the requested one is not suitable. For instance, an encoder
device may decide that it needs more larger output buffers in order to
encode at the requested format and resolution.

\drivernormative{\paragraph}{Format parameters}{Device Types / Video Device / Parameters / Format parameters}

When setting a format parameter, the driver MUST check the adjusted
returned value and comply with it, or try to set a different one if it
cannot.

\subsubsection{Encoder parameters}\label{sec:Device Types / Video Device / Parameters / Encoder parameters}

\begin{lstlisting}
struct virtio_video_params_bitrate {
    le32 min_bitrate;
    le32 max_bitrate;
    le32 bitrate;
    u8 padding[4];
}
\end{lstlisting}

\begin{description}
\item[\field{min_bitrate}]
is the minimum bitrate supported by the encoder for the current
settings. Ignored when setting the parameter.
\item[\field{max_bitrate}]
is the maximum bitrate supported by the encoder for the current
settings. Ignored when setting the parameter.
\item[\field{bitrate_}]
is the current desired bitrate for the encoder.
\end{description}

\subsection{Supported formats}\label{sec:Device Types / Video Device / Supported formats}

Bitstream and image formats are identified by their fourcc code, which
is a four-bytes ASCII sequence uniquely identifying the format and its
properties.

\subsubsection{Bitstream formats}\label{sec:Device Types / Video Device / Supported formats / Bitstream formats}

The fourcc code of each supported bitstream format is given, as well as
the unit of data requested in each input resource for the decoder, or
produced in each output resource for the encoder.

\begin{description}
\item[\field{MPG2}]
MPEG2 encoded stream. One Access Unit per resource.
\item[\field{H264}]
H.264 encoded stream. One NAL unit per resource.
\item[\field{HEVC}]
HEVC encoded stream. One NAL unit per resource.
\item[\field{VP80}]
VP8 encoded stream. One frame per resource.
\item[\field{VP90}]
VP9 encoded stream. One frame per resource.
\end{description}

\subsubsection{Image formats}\label{sec:Device Types / Video Device / Supported formats / Image formats}

The fourcc code of each supported image format is given, as well as its
number of planes, physical buffers, and eventual subsampling.

\begin{description}
\item[\field{RGB3}]
one RGB plane where each component takes one byte, i.e.~3 bytes per
pixel.
\item[\field{NV12}]
one Y plane followed by interleaved U and V data, in a single buffer.
4:2:0 subsampling.
\item[\field{NV12}]
same as \field{NV12} but using two separate buffers for the Y and UV
planes.
\item[\field{YU12}]
one Y plane followed by one Cb plane, followed by one Cr plane, in a
single buffer. 4:2:0 subsampling.
\item[\field{YM12}]
same as \field{YU12} but using three separate buffers for the Y, U and V
planes.
\end{description}
